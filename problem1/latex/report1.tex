\documentclass[12pt]{article}

\usepackage[utf8]{inputenc}
\usepackage[russian]{babel}
\usepackage{amsmath, amssymb}
\usepackage{graphicx}
\usepackage{listings}

\lstset{
	language=Python,
	keepspaces=true,
	extendedchars=\true,
	inputencoding=utf8,
	basicstyle=\tiny,
	numbers=left,
	showspaces=false,
	showstringspaces=false,
}

\def\dd#1#2{\cfrac{\partial#1}{\partial#2}}

\author{Будакян Я. С.}

\begin{document}

	\begin{titlepage}
	
		\begin{center}
			{\small\textsc{Московский Государственный Университет им. М.\,В. Ломоносова}}
			\vskip 1pt \hrule \vskip 3pt
			{\small\textsc{Физический факультет}}
			\vfill
			{\Large Практическое задание по ОММ}
			\break
			\break
			{\Large Задача \#1}	
		\end{center}
		\vfill
		\begin{flushright}
			{Выполнил студент 335 группы \\Будакян Я.\,С.}
		\end{flushright}
	\end{titlepage}
	
	\section{Постановка задачи}
		\bigskip\par\noindent{\bf Задача 2. }
		Используя схему бегущего счета и итерационные методы, решить задачу:
		\begin{equation}\label{eq:problem}
			\begin{cases}
				&\dd{u}t - u\dd{u}x = 0, \quad -1 \le x < 0, \\
				&u(x, 0) = 2 - \frac{4}{\pi}\arctg(x+2), \\
				&u(0, t) = (2 - \frac{4}{\pi}\arctg2)e^{-t}
			\end{cases}
		\end{equation}
	
	\section{Метод решения}
		
	
\end{document}
